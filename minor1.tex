\documentclass[12pt]{article}

\textwidth6.75in \textheight9in \oddsidemargin-10pt \evensidemargin-10pt
\topmargin-47pt



\usepackage{amsmath}
\usepackage{amsfonts}
\usepackage{amscd}
\usepackage{epsfig}
\usepackage{amssymb} 



\newtheorem{theorem}{Theorem}[section]
\newtheorem{definition}{Definition}[section]
\newtheorem{problem}{Problem}[section]
\def\rta{\stackrel{a}{\rightarrow}}
\def\rtb{\stackrel{b}{\rightarrow}}
\def\CR{{\cal R}}
\def\CA{{\cal A}}
\def\CP{{\cal P}}
\def\CI{{\cal I}}
\def\CT{{\cal T}}
\def\ex{\mbox{E}}
\def\pr{\mbox{P}}
\def\tmix{t_{\mbox{mix}}}

\newcommand{\NN}{\mathbb{N}}
\newcommand{\RR}{\mathbb{R}}
\newcommand{\LL}{\mathbb{L}}
\newcommand{\ZZ}{\mathbb{Z}}

\begin{document}
\begin{center}
{\large COL863: Special Topics in Theoretical Computer Science\\ Rapid Mixing in Markov Chains\\ II semester, 2016-17}\\
{\large Minor I}\\
Total Marks 100\\
{\bf Due: On moodle at 11:55AM, 6th February 2017}
\ \\
\end{center}

%\maketitle
\setcounter{section}{1}

\begin{problem} {\bf (20 marks)}
Ex 4.4 of LPW Edition 2.
\end{problem}

\centerline{\rule{0.75\paperwidth}{0.4pt}}
We know that $\exists$ a coupling $(X,Y) \| \mathbb{P}[X\neq Y] = \lVert\mu-\nu\rVert_{TV} \forall \mu,\nu$ \\ Specifically, let $(X^i,Y^i)$ be such a coupling for $\mu^i,\nu^i$. Since each $X^i$ is independent and so is each $Y^i$, we can define $(\textbf{X}\equiv X^1,X^2...,X^n, \textbf{Y}\equiv Y^1,Y^2...,Y^n)$ as a general coupling on $\mu,\nu$. \\
We also know that $\forall$ couplings $(X,Y) on \mu,\nu$, $\lVert\mu-\nu\rVert_{TV} \leq \mathbb{P}[X\neq Y]$ \\
Hence, $$\lVert\mu-\nu\rVert_{TV} \leq \mathbb{P}[X\neq Y]$$ 
$$\mathbb{P}[\textbf{X}\neq \textbf{Y}] \leq \sum_{i=0}^{n} \mathbb{P}[X_i \neq Y_i]$$ 
$$ \sum_{i=0}^{n} \mathbb{P}[X\neq Y] = \sum_{i=0}^{n} \lVert\mu-\nu\rVert_{TV} $$
All that is left to show is that $(\textbf{X},\textbf{Y})$ is a valid coupling. 
$$\mathbb{P}[\textbf{X} = x] = \sum_{\textbf{Y}}\mathbb{P}[\textbf{X}=x,\textbf{Y}] $$
$$ = \sum_{\textbf{Y}} \prod_{j=0}^{n} \Pr[X_j=x_j,Y_j]$$
$$ = \prod_{j=0}^{n} \sum_{\textbf{Y}} \Pr[X_j=x_j,Y_j]$$
$$ = \prod_{j=0}^{n} \mu_j $$
$$ = \mu $$

\vspace{1cm}
\centerline{\rule{0.75\paperwidth}{0.4pt}}



\begin{problem} {\bf (40 marks)}
You have a deck that contains $n$ cards. The shuffling process is 
\begin{quote}
Make two independent choices of cards and interchange them.
\end{quote}
Note that the two choices may be the same, in which case there is no change in the configuration of the deck. Use a coupling based method to prove an upper bound on $\tmix(\epsilon)$ for this chain.
\end{problem}

\centerline{\rule{0.75\paperwidth}{0.4pt}}
{\sf **** REPLACE THIS WITH YOUR ANSWER ****}

\vspace{1cm}
\centerline{\rule{0.75\paperwidth}{0.4pt}}

\begin{problem} {\bf (40 marks)}
Again: You have a deck that contains $n$ cards. This time let us call the $n$ positions in the deck $0, 1, \ldots, n-1$. The (lazy) shuffling process is 
\begin{quote}
With probability 1/2 do nothing. Otherwise pick $i$ uniformly at random from $0, 1, \ldots, n-1$ and interchange the cards in positions $i$ and $(i+1) \mod n$.
\end{quote}
Use a coupling based method to prove an upper bound on $\tmix(\epsilon)$ for this chain.
\end{problem}

\centerline{\rule{0.75\paperwidth}{0.4pt}}
{\sf **** REPLACE THIS WITH YOUR ANSWER ****}

\vspace{1cm}
\centerline{\rule{0.75\paperwidth}{0.4pt}}



\end{document}
